\specialite{Exemple}

%%%%%%%%%%%%%%%%%%%%%%%%%%%%%
% LES ATTENDUS
%%%%%%%%%%%%%%%%%%%%%%%%%%%%%

% Si aucun attendu n'est mis, les cases sont simplement ignorées
% Les lignes commençant par un signe % sont ignorées aussi

\nouvelattendugeneral{Être capable de conceptualisation et d’abstraction}
\nouvelattendugeneral{Avoir une maitrise du français permettant de ...}
\nouvelattendugeneral{Avoir une connaissance suffisante de l’anglais permettant de ...}
\nouvelattendutech{Bien connaître toutes les techniques de (domaine spécifique)}
\nouvelattendutech{Savoir se servir d’une machine à café}
\nouvelattenduhumain{Avoir l'esprit d'équipe}
\nouvelattenduhumain{Savoir s'intégrer dans les travaux de groupe}

%%%%%%%%%%%%%%%%%%%%%%%%%%%%%
% LES PARCOURS
%%%%%%%%%%%%%%%%%%%%%%%%%%%%%

\nouveauparcours{Développement de bidule}
\nouveaumetieraccessible{Biduliste (exemple de métier niveau bac+3 lié au parcours)}
\nouveaumetieraccessible{Développeur-biduliste (exemple de métier niveau bac+3 lié au parcours)}
\nouveaumetiersenior{Expert-biduliste (métier senior accessible après le parcours)}
\nouveaumetierpossible{Assistant-chosiste (métier éventuellement possible après le parcours)}
\nouveaumetierpossible{Chargé de truc (métier éventuellement possible après le parcours)}
\nouveaumetierpossible{Machining Technician (n'utiliser les noms en anglais que si pas de d'équivalent français)}

\nouveauparcours{Développement de machin}
\nouveaumetieraccessible{Biduliste appliqué à l'autre machin}
\nouveaumetieraccessible{Machinisteur}
\nouveaumetieraccessible{Assistant-machiniste}
\nouveaumetierpossible{Machino-biduleur}
\nouveaumetierpossible{Chargé de truc}
\nouveaumetiersenior{Machiniste}

%%%%%%%%%%%%%%%%%%%%%%%%%%%%%
% LES DOMAINES DE RESSOURCES
%%%%%%%%%%%%%%%%%%%%%%%%%%%%%

% Si tous les domaines sont commentés, il n'y aura aucun affichage de ressources

\nouveaudomaine{Manuel}
\nouveaudomaine{Écriture}
\nouveaudomaine{Technicité}
\nouveaudomaine{Comm.--LV}
\nouveaudomaine{PPP}

% ATTENTION, TOUTE MODIFICATION DE L'ORDRE OU DU NOMBRE DES RESSOURCES CI-DESSUS IMPLIQUE DE MODIFIER
% LES MATRICES D'APPRENTISSAGE CRITIQUE

%%%%%%%%%%%%%%%%%%%%%%%%%%%%%
% Les blocs de compétences : définitions et niveaux
%%%%%%%%%%%%%%%%%%%%%%%%%%%%%

% Dans le vocable de la note de cadrage, ce sont en fait les activités que l'on met ici
% C'est un nom. La première définition comprend des verbes et est la compétence proprement dite.
%
% NB: on peut rajouter une deuxième définition qui sera mise derrière la première
% par exemple :
% \ajoutdef{F}{\textbf{Compétence transversale} à tous les BUT.}
% Si on ne met pas de nom de niveau (l'espace vide {}), les noms par défaut sont donnés :
% Niveau 1 / Niveau 2 / Niveau 3


\nouvellecompetence{A}{Création de bidule}{Créer des bidules pour faire des choses et des trucs}
\ajoutniveau{A}{}{Comprendre les relations entre bidules}
\ajoutniveau{A}{}{Intégrer les bidules dans les contextes chosiques}
\ajoutniveau{A}{}{Créer des bidules pour des extra-choses et des trucs de spécialistes}
\competenceparcoursniveau{A}{A}{A,B,C}
\competenceparcoursniveau{A}{B}{A,B}

\nouvellecompetence{B}{Compétence autre}{Faire des choses et des machins qui constituent l'autre compétence}
\ajoutniveau{B}{}{blablabla}
\ajoutniveau{B}{}{blablabla}
\ajoutniveau{B}{}{blablabla}
\competenceparcoursniveau{B}{A}{A,,B}
\competenceparcoursniveau{B}{B}{A,B,C}

\nouvellecompetence{C}{Compétence tierce}{Constituer des machins-choses et des trucs}
\ajoutniveau{C}{}{blablabla}
\ajoutniveau{C}{}{blablabla}
\ajoutniveau{C}{}{blablabla}
\competenceparcoursniveau{C}{A}{A,B,C}
\competenceparcoursniveau{C}{B}{,A,B}

%%%%%%%%%%%%%%%%%%%%%%%%%%%%%
% Les composantes essentielles
%%%%%%%%%%%%%%%%%%%%%%%%%%%%%

% ce sont des participes présents

\ajoutcompo{A}{adaptant des solutions aux besoins des clients}
\ajoutcompo{A}{respectant la législation, les normes professionnelles et les enjeux sociétaux}
\ajoutcompo{A}{comprenant que chaque composante essentielle doit se traduire par une partie de rapport dans chaque SAÉ}

\ajoutcompo{B}{trouvant des composantes essentielles}
\ajoutcompo{B}{respectant des enjeux de bidules}

% %%%%%%%%%%%%%%%%%%%%%%%%%%%%% 
% % Les situations professionnelles
% %%%%%%%%%%%%%%%%%%%%%%%%%%%%%

\ajoutsituation{A}{Élaborer un bidule pour animaux}
\ajoutsituation{A}{Élaborer un bidule pour cailloux}

\ajoutsituation{B}{Créer un machin pour liquide}
\ajoutsituation{B}{Créer un machin pour solide}
\ajoutsituation{B}{Fusionner des machins pré-existants}

% %%%%%%%%%%%%%%%%%%%%%%%%%%%%% 
% % Les apprentissages critiques
% %%%%%%%%%%%%%%%%%%%%%%%%%%%%%
% % Premier argument : identifiant du bloc de compétence (A,B,...)
% % Deuxième argument : identifiant du niveau (A,B,C)
% % Troisième argument : Description
% % Quatrième argument : une liste de 0 et 1 séparés par des virgules pour chaque participation
% %                      d'un domaine de ressource (dans l'ordre ci-dessus) à un apprentissage

% \ajoutapprentissage{A}{A}{Implémenter des conceptions simples}{1,0,0,1,0,0,1,0}
% \ajoutapprentissage{A}{A}{Élaborer des conceptions simples}{0,1,1,0,1,0,1,0}
% \ajoutapprentissage{A}{A}{Faire des essais et évaluer leurs résultats en regard des spécifications}{1,1,0,1,1,0,0,0}
% \ajoutapprentissage{A}{A}{Développer des interfaces utilisateurs}{1,1,0,0,1,1,0,0}
% \ajoutapprentissage{A}{B}{Élaborer et implémenter les spécifications fonctionnelles et non fonctionnelles à partir des exigences}{1,1,0,1,1,1,0,0}
% \ajoutapprentissage{A}{B}{Appliquer des principes d’ergonomie et d’accessibilité}{1,1,0,0,0,0,0,0}
% \ajoutapprentissage{A}{B}{Adopter de bonnes pratiques de conception et de programmation}{1,1,0,0,0,1,0,0}
% \ajoutapprentissage{A}{B}{Utiliser des patrons de conception pour le développement d’applications cohérentes}{1,1,0,0,0,0,0,0}
% \ajoutapprentissage{A}{B}{Adapter les solutions existantes au contexte applicatif}{1,1,0,0,1,1,0,0}
% \ajoutapprentissage{A}{B}{Vérifier et valider la qualité de l’application par les tests}{1,1,0,0,1,1,0,0}
% \ajoutapprentissage{A}{C}{Choisir et implémenter les architectures adaptées}{1,1,0,0,0,0,0,0}
% \ajoutapprentissage{A}{C}{Développer des applications sur des supports spécifiques}{1,1,0,0,0,1,0,0}
% \ajoutapprentissage{A}{C}{Réaliser un audit d’une application}{0,1,0,0,1,0,0,0}
% \ajoutapprentissage{A}{C}{Intégrer des solutions dans un environnement de production}{1,0,1,0,0,1,0,0}

% \ajoutapprentissage{B}{A}{Mettre à jour et interroger une base de données relationnelle (en requêtes directes ou à travers un ORM)}{0,0,0,1,1,1,1,0}
% \ajoutapprentissage{B}{A}{Visualiser des données}{0,0,0,1,0,1,1,0}
% \ajoutapprentissage{B}{A}{Concevoir une base de données relationnelle à partir d’un cahier des charges}{0,0,0,1,1,1,1,0}
% \ajoutapprentissage{B}{B}{Modéliser et structurer les données de l’entreprise}{0,0,0,1,1,1,1,0}
% \ajoutapprentissage{B}{B}{Assurer l’intégrité et la confidentialité des données}{0,0,0,1,1,1,1,0}
% \ajoutapprentissage{B}{B}{Organiser la restitution de données à travers la programmation et la visualisation}{0,0,0,1,1,1,1,0}
% \ajoutapprentissage{B}{C}{Capturer et stocker des ensembles volumineux et complexes de données hétérogènes}{0,0,0,1,0,1,1,0}
% \ajoutapprentissage{B}{C}{Préparer les données pour l’exploitation}{0,0,0,1,0,1,1,0}
% \ajoutapprentissage{B}{C}{Appliquer des méthodes d’exploration et d’exploitation des données (apprentissage, informatique décisionnelle ou fouille de données)}{0,0,0,1,0,1,1,0}
% \ajoutapprentissage{B}{C}{Mettre en production et optimiser le système de gestion de données de l’entreprise}{0,0,0,1,0,1,1,0}

% \ajoutapprentissage{C}{A}{Identifier les différents composants (matériels et logiciels) d’un système numérique}{0,0,1,0,0,0,1,0}
% \ajoutapprentissage{C}{A}{Utiliser les fonctionnalités de base d’un système multitâches / multiutilisateurs}{0,0,1,0,0,0,0,0}
% \ajoutapprentissage{C}{A}{Installer et configurer un système d’exploitation et des outils de développement}{0,0,1,0,0,0,0,0}
% \ajoutapprentissage{C}{A}{Configurer un poste de travail dans un réseau d’entreprise}{0,0,1,0,0,1,0,0}
% \ajoutapprentissage{C}{A}{Lire une documentation technique (en anglais)}{0,0,1,0,1,0,0,0}
% \ajoutapprentissage{C}{B}{Concevoir et développer des applications communicantes}{1,1,1,0,0,0,0,0}
% \ajoutapprentissage{C}{B}{Utiliser des serveurs et des services réseaux virtualisés}{1,1,1,0,0,0,0,0}
% \ajoutapprentissage{C}{B}{Sécuriser les serices et données d’un système}{1,0,1,0,0,1,1,0}
% \ajoutapprentissage{C}{B}{Rédiger une documentation technique (en anglais)}{0,0,1,0,1,1,0,0}
% \ajoutapprentissage{C}{C}{Créer des processus de traitement automatisé (écriture de scripts, solution de gestion de configuration et de parc, intégration et déploiement continu...)}{1,1,1,0,1,1,0,0}
% \ajoutapprentissage{C}{C}{Configurer un serveur et des services réseaux de manière avancée (virtualisation, ...)}{0,1,1,0,1,1,0,0}
% \ajoutapprentissage{C}{C}{Appliquer une politique de sécurité au niveau de l’infrastructure}{0,0,1,1,1,1,0,0}
% \ajoutapprentissage{C}{C}{Déployer un réseau d'organisation en fonction de ces besoins }{1,1,1,1,1,1,1,0}
% \ajoutapprentissage{C}{C}{Maintenir l’infrastructure en condition opérationnelle (sauvegarde, supervision , haute disponibilité....)}{1,1,1,1,1,1,0,0}
