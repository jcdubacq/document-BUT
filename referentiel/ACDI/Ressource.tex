% # context : data, utils, ress
%% set year=utils.semestre2year(ress.getSemestreList()[0])|le
%% set possiblecomps=data.getPossibleCompObjects(ress.getParcoursObjects(),year)
%% set complettres=['X','A','B','C','D','E','F']
\newpage
\invisiblesubsubsection{ANNEXE D --- fiche Ressource \VAR{ress.getId()|le}~\emph{\VAR{ress.getShorttxt()|le}}}\label{FICHE-\VAR{ress.getId()|le}}
\selectparcours{\VAR{data.getParcoursCanonical(ress.getParcoursObjects())|le}}
\begin{tikzpicture}[remember picture,overlay,x=1mm,y=1mm,every node/.style={inner sep=0pt,outer sep=0pt}]
  \coordinate (NW) at (2,42);
  \coordinate (NE) at (188,42);
  \coordinate (NWi) at (10,42);
  \coordinate (NEi) at (180,42);
  \coordinate (NW) at (current page.north west);
  \coordinate (NE) at (current page.north east);
  \coordinate (NWi) at ($ (NW)+(18,0) $);
  \coordinate (NEi) at ($ (NE)+(-18,0) $);
  \coordinate (start) at ($ (current page.north west)+(0,-40) $);
  \coordinate (mark5) at ($ 1/2*(NWi)+1/2*(NEi)+1/2*(-10,0) $);
  \path ($ (mark5)-(NWi) $);\pgfgetlastxy{\xmarkfive}{\ymarkfive};
  \coordinate (mark4) at ($ 2/3*(NWi)+1/3*(NEi)+1/3*(-16,0) $);
  \coordinate (mark6) at ($ 1/3*(NWi)+2/3*(NEi)+1/3*(-14,0) $);
  \coordinate (mark0) at ($ (NEi)+(-4,0) $);
  \path ($ (mark0)-(NWi) $);\pgfgetlastxy{\xmarkzero}{\ymarkzero};
  % 
  % Title
  % 
  \draw[black,line width=1mm] (start-|NW)--(start-|NWi);
  \draw[black,line width=1mm] ($ (start-|NW)+(0,10) $)--($ (start-|NWi)+(0,10) $);
  \node[draw,black,fill=black,text=white,line width=1mm,rounded corners=5mm,text width=30mm,minimum height=30mm,anchor=south west,inner sep=0pt,outer sep=0pt] (bigtitle) at ($ (start-|NWi)+(0,-10) $) {\centering\LARGE\bfseries\selectfont Ressource\\\vspace*{2mm}\fontsize{32}{32}\bfseries\selectfont \VAR{ress.getId()|le}};
  \draw[black,line width=1mm] ($ (start-|NWi)+(30,0) $)--(start-|NE);
  \draw[black,line width=1mm] ($ (start-|NWi)+(30,10) $)--($ (start-|NE)+(0,10) $);
  \node[anchor=west,color=blacktextemphcolor,text height=3ex,text depth=1ex] at ($ (start-|NWi)+(32,5) $) {\normalfont\fontsize{13}{13}\bfseries\color{blacktextemphcolor}\VAR{ress.getLongtxt()|le}};
  %
  % Breadcrumbs
  %
  %% set arianeList=ress.getDisciplineList()
  \coordinate (last) at (NEi|-start);
  %% for ariane in arianeList
  \node[anchor=north east] (ariane) at ($ (last)+(0,-1) $) {%
    \normalfont\fontsize{9}{9}\bfseries\color{blacktextdimmedcolor}%
    %% for piece in ariane
    \VAR{piece|le}
    %% if loop.last
    \color{blacktextdimmedcolor!50!whitepagecolor}
    %% endif
    >
    %% else
    \color{blacktextdimmedcolor!50!whitepagecolor}    
    %% endfor
    \VAR{ress.getShorttxt()|le}%
  };
  \coordinate (last) at (ariane.south east);
  %% endfor
  \coordinate (arianesouth) at (last);
  %% set module=ress
  %% include 'Fresco.tex'
  %
  % Boxes
  % 
  \node[fit=(bigtitle.south)(arianesouth)] (bottom) {};
  \coordinate (last) at ($ (bottom.south)+(0,-3) $);
  \coordinate (lasteast) at (last-|NEi);
  \coordinate (lastwest) at (last-|NWi);
  \node[anchor=north west,text width={\xmarkzero}] (titredescriptif) at ($ (last-|NWi)+(2,-2) $) {
    \sloppy\bfseries\large\color{blacktextemphcolor}\selectfont
    Descriptif détaillé
  };
  \coordinate (last) at (titredescriptif.south-|NWi);
  \node[anchor=north west,text width={\xmarkzero}] (descriptif) at ($ (last)+(2,-1) $) {
    \setlength\topsep{0pt}\setlength\partopsep{0pt}\begin{flushleft}
      %% if ress.getObjectifList()
      \textbf{Quel objectif pour cette ressource ?}\\
      \VAR{ress.getObjectifList()|le|join('\\par\n')}\par
      %% endif
      %% set sav=ress.getSavoirsList()
      %% if sav|length>0
      \textbf{Quels savoirs de référence à étudier ?}\\
      \begin{itemize}[nosep,topsep=0pt,label=\textitemize,leftmargin=1pc,labelsep=*]
        %% for s in sav
      \item \VAR{s|le}
        %% endfor
      \end{itemize}
      %% endif
      %% if ress.getDescriptionList()
      \textbf{Comment cette ressource fait monter en compétence ?}\\    
      \VAR{ress.getDescriptionList()|le|join('\\par\n')}
      %% endif
    \end{flushleft}
  };
  \coordinate (last) at (descriptif.south-|NWi);
  %% set keywords=ress.getKeywordsList()|join(',')
  %% if keywords|length > 0
  \node[anchor=north west,text width={\xmarkzero}] (keywords) at ($ (last)+(2,0) $) {
    \begin{center}
      %% for k in keywords.split(',')
      \tcbox[colback=whitepagecolor, colframe=blackpagecolor, on line, boxsep=0pt, left=4pt, right=4pt, top=4pt, bottom=4pt]{\textcolor{blacktextcolor}{\rule[-1ex]{0mm}{3ex}\VAR{k|trim|le}}}
      %% endfor
    \end{center}
  };
  %% else
  \coordinate (keywords) at (last);
  %% endif
  \coordinate (last) at ($ (keywords.south)+(0,-4) $);
  \coordinate (lasteast) at (last-|NEi);
  \coordinate (lastwest) at (last-|NWi);
  
  \node[%
  inner sep=2mm,rounded corners=2mm,%
  line width=.5mm,draw=black,%
  text width={\xmarkfive},anchor=north west] (combobox) at (lastwest-|NWi) {%
    {\large\bfseries Cursus}\hfill S\VAR{ress.getSemestreList()|elegantjoin|le}\\
    \textbf{Prérequis :}
    %% for prq in ress.getPrerequisObjects()
    \hyperref[FICHE-\VAR{prq.getId()|le}]{\VAR{prq.getId()|le}~\VAR{prq.getShorttxt()|le}}
    %% if not loop.last
    ;
    %% endif
    %% else
    aucun
    %% endfor
    \\
    %% set allsaes=data.getSAEObjects()
    %% set portfolios=data.getPortfolioObjects()
    %% set source=data.getHoursBlock(ress.getId())
    \textbf{Heures totales}\dotfill%
    \textcolor{blacktextemphcolor}{\bfseries\VAR{source.sumIn(infotype=['TD'])|hours}} TD et 
    \textcolor{blacktextemphcolor}{\bfseries\samebox{\bfseries 00h}{\VAR{source.sumIn(infotype=['TP'])|hours}}} TP\\[.5ex]
    \emph{dont ressource}\dotfill
    \VAR{(source.sumIn(infotype=['TD'])-source.sumOut(infotype=['TD']))|hours} TD et 
    \samebox{\bfseries 00h}{\VAR{(source.sumIn(infotype=['TP'])-source.sumOut(infotype=['TP']))|hours}} TP%
    \llap{\rule[2.1ex]{8em}{1pt}}
    \\
    %% for sae in allsaes if ress in sae.getRessourceObjects() or source.sumOut(destination=[sae.getId()])>0
    %% if loop.first
    Lien\VAR{'s' if loop.length>1 else ''} avec les SAÉ :\\
    %% endif
    \hyperref[FICHE-\VAR{sae.getId()|le}]{\VAR{sae.getId()|le}~\VAR{sae.getShorttxt()|le}}
    %% if source.sumOut(destination=[sae.getId()])>0
    \dotfill
    \VAR{source.sumOut(destination=[sae.getId()],infotype=['TD'])|hours} TD et 
    \samebox{\bfseries 00h}{\VAR{source.sumOut(destination=[sae.getId()],infotype=['TP'])|hours}} TP
    %% endif
    \\
    %% endfor
    %% for sae in portfolios if ress in sae.getRessourceObjects() or source.sumOut(destination=[sae.getId()])>0
    %% if loop.first
    Lien\VAR{'s' if loop.length>1 else ''} avec le portfolio :\\
    %% endif
    \hyperref[FICHE-\VAR{sae.getId()|le}]{\VAR{sae.getId()|le}~\VAR{sae.getShorttxt()|le}}
    %% if source.sumOut(destination=[sae.getId()])>0
    \dotfill
    \VAR{source.sumOut(destination=[sae.getId()],infotype=['TD'])|hours} TD et 
    \samebox{\bfseries 00h}{\VAR{source.sumOut(destination=[sae.getId()],infotype=['TP'])|hours}} TP
    %% endif
    \\
    %% endfor
  };

  \coordinate (lastwest) at ($ (combobox.south-|NWi)+(0,-2) $);

  %% for comp in possiblecomps if comp.getCoeff(ress) or ress.getACObjects(comp=comp)
  %% set compnum=comp.getNum()|int
  %% set compltr=complettres[compnum]
  %% set acs=ress.getACObjects(comp=comp)
  %% set coeffs=comp.getCoeff(ress)
  %% set sem=ress.getSemestreList()[0]
  %% set coeffsum=comp.getCoeffs(onlyType=['RESS','PORTFOLIO','SAE'],semestre=sem)
  %% set coeffsumstr=('/'+coeffsum[1]) if coeffsum[1]!=100 else '%'
  \compareytozero{($ (lasteast)-(lastwest) $)}{\coordinate(stpoint) at (lastwest);}{\coordinate(stpoint) at ($ (lasteast)-(mark5)+(NWi)+(-4,0) $);}
  \node[text=comp\VAR{compltr}pcolor,fill=comp\VAR{compltr}color,%
  inner sep=2mm,rounded corners=2mm,%
  line width=.5mm,draw=comp\VAR{compltr}color,%
  text width={\xmarkfive},anchor=north west] (ac\VAR{compnum}title) at (stpoint) {%
    \Large\bfseries
    %% if comp.isPlural()>1
    \VAR{comp.getLongtxt()}
    %% else
    Compétence \VAR{compnum}
    %% endif
  };
  %% if coeffs
  \node[draw=blackpagecolor,inner sep=1mm,anchor=north east,fill=whitepagecolor,rounded corners] (coeffs) at ($ (ac\VAR{compnum}title.north east)+(-1,-1) $)
  {\textcolor{blacktextcolor}{\large\bfseries\VAR{coeffs[1]|le}\VAR{coeffsumstr|le}}};
  %% endif

  \coordinate (last) at (ac\VAR{compnum}title.south west);
  \node[text=blacktextcolor,fill=comp\VAR{compltr}color!25!whitepagecolor,%
  inner sep=2mm,%
  text width={\xmarkfive},anchor=north west] (ac\VAR{compnum}level) at ($ (last)+(0,1.5) $) {%
    \setlength\topsep{0pt}\setlength\partopsep{0pt}\begin{flushleft}
      %% if comp.isPlural()>1
      Enseignement transversal
      %% else
      \VAR{comp.getNiveauDict(year).values()|elegantjoin|le}
      %% endif
    \end{flushleft}
  };
  \coordinate (last) at (ac\VAR{compnum}level.south west);
  \node[text=blacktextcolor,fill=comp\VAR{compltr}color!25!whitepagecolor,%
  inner sep=2mm,%
  text width={\xmarkfive},anchor=north west] (ac\VAR{compnum}acs) at ($ (last)+(0,1.5) $) {
    \setlength\topsep{0pt}\setlength\partopsep{0pt}\begin{flushleft}
      %% if acs
      %% for ac in acs
      %% if not ac.isPlural()>1
      \textbf{AC \VAR{ac.getNum()|le}}
      %% endif
      \VAR{ac.getLongtxt()|le}
      %% if not loop.last
      \par
      %% endif
      %% endfor
      %% else
      (pas d'apprentissage critique)
      %% endif
    \end{flushleft}
  };
  \draw[line width=.5mm,draw=comp\VAR{compltr}color] (ac\VAR{compnum}title.base west)--(ac\VAR{compnum}acs.south west)--(ac\VAR{compnum}acs.south east)--(ac\VAR{compnum}title.base east);
  \draw[line width=.5mm,draw=comp\VAR{compltr}color,dotted] ($ (0,-1)+(ac\VAR{compnum}acs.north west) $)--($ (0,-1)+(ac\VAR{compnum}acs.north east) $);
  \compareytozero{($ (lasteast)-(lastwest) $)}{\coordinate(lastwest) at ($ (ac\VAR{compnum}acs.south west)+(0,-2) $);}{\coordinate(lasteast) at ($ (ac\VAR{compnum}acs.south east)+(0,-2) $);}
  %% endfor

  \begin{pgfonlayer}{bg}
    \node[inner sep=2mm,rounded corners=2mm,draw=blacktextcolor,line width=.5mm,fit=(descriptif) (titredescriptif) (keywords)]{};
  \end{pgfonlayer}



\end{tikzpicture}
\vfill