\documentclass[10pt]{article}

% All encoding stuff
\usepackage[LUC,T2A,T1]{fontenc}%
\usepackage[verbose]{newunicodechar}
\usepackage{fontspec}
\usepackage[T1]{fontenc}
\usepackage{textcomp}

% Drawing stuff
\usepackage[usenames,dvipsnames]{xcolor}
\usepackage[prefix=sol-]{xcolor-solarized}
\usepackage{tikz}%
\usetikzlibrary{tikzmark,calc,fit,shapes,shapes.misc,arrows,shadows.blur}%
\usepackage{tcolorbox}
\usepackage{graphicx}
\usepackage{pdfpages}
\pgfdeclarelayer{bg}    % declare background layer
\pgfsetlayers{bg,main}

%% include 'Color.tex'


% Typography stuff
\usepackage{geometry}
\usepackage[headings]{fancyhdr}
\usepackage{dcolumn,colortbl}
\usepackage{soul} % better underline \ul
\setuldepth{Berlin} % no descending to imitate word

% Linking stuff

\usepackage{hyperref,bookmark}
\hypersetup{%
  colorlinks,
  urlcolor=blacktextemphcolor,
  linkcolor=blacktextcolor
}

% Pure LaTeX stuff

\usepackage{ifthen}
\usepackage{tabularx,multirow,booktabs,ragged2e}
\usepackage{colortbl}%
\usepackage[french]{babel}
\usepackage{enumitem,xspace}
\newcommand\samebox[2]{%
  \makebox[\widthof{#1}][r]{#2}%
}

\newlength{\mywidth}
\newcommand{\resetwidth}{\setlength{\mywidth}{0pt}}
\newcommand{\maxwidth}[1]{\setlength{\mywidth}{\maxof{\mywidth}{\widthof{#1}}}}
\newcommand{\rothead}[1]{\rotatebox[origin=l]{90}{\parbox{\mywidth}{#1}}}

\newcommand{\compareytozero}[3]{\path #1;\pgfgetlastxy{\xdelta}{\ydelta};\ifdim\ydelta<0pt#2\else#3\fi}
\setcounter{tocdepth}{2}
\newcolumntype{H}{r}

% Unused stuff
% \usepackage[utf8x]{inputenc}%
% \usepackage{etoolbox}
% \usepackage[scale=.5,text=DOCUMENT~DE~TRAVAIL]{draftwatermark}

% Typographic settings

\makeatletter
\newcommand\HUGE{\@setfontsize\Large{32pt}{40}}
\renewcommand\Huge{\@setfontsize\Large{24pt}{30}}
\renewcommand\LARGE{\@setfontsize\Large{16pt}{20}}
\renewcommand\Large{\@setfontsize\Large{13pt}{16}}
\renewcommand\large{\@setfontsize\Large{11pt}{13}}
\renewcommand\normalsize{\@setfontsize\Large{10pt}{12}}
\renewcommand\small{\@setfontsize\Large{9pt}{11}}
\renewcommand\footnotesize{\@setfontsize\Large{8pt}{9}}
\makeatother

\geometry{a4paper,left=10mm,right=10mm,top=25 mm,bottom=25mm}
\setmainfont{Arial}
\makeatletter
\newunicodechar{«}{\fontspec{Arial}\char"00AB\FB@guillspace\ignorespaces}
\newunicodechar{»}{\unskip\FB@guillspace\fontspec{Arial}\char"00BB}
\newunicodechar{©}{\fontspec{Arial}\char"00A9\unskip\,}
\makeatother
\setlength{\parindent}{0pt}
\setlength{\tabcolsep}{2mm}\setlength{\arrayrulewidth}{.2mm}

\pagestyle{headings}
\renewcommand{\sectionmark}[1]{%
  \gdef\currsection{#1}%
}
\renewcommand{\subsectionmark}[1]{%
  \markright{\currsection\ \textbullet\ #1}%
}
\def\fancy{%
  \fancyhf{}%
  \fancyhead[L]{\small\emph{\rightmark}}%
  \fancyhead[R]{\thepage}%
  \fancyfoot[L]{\textcolor{blacktextdimmedcolor}{©~\VAR{data.author|le}}\\%
    \href{\VAR{data.url|le}}{\ul{\VAR{data.url|le}}}}%
  \fancyfoot[R]{\textcolor{blacktextdimmedcolor}{\VAR{data.version|le}\\\VAR{data.getName()|le} (\leparcours)}}%
}
\fancy

\renewcommand{\headrulewidth}{0pt}

\newcommand{\selectallparcours}{\def\leparcours{\VAR{data.getParcours().getCanonical()|le}}}
\newcommand{\selectparcours}[1]{\gdef\leparcours{#1}}
\selectallparcours

% Only three sizes of titles (including normalsize)
\usepackage{titlesec}
\renewcommand\thesection{\Alph{section}}
\renewcommand\thesubsection{\arabic{subsection}}
\renewcommand\thesubsubsection{\arabic{subsubsection}}
\newcommand\invisiblesubsubsection[1]{\refstepcounter{subsubsection}  \addcontentsline{toc}{subsubsection}{\protect\numberline{\thesubsubsection}#1} \subsubsectionmark{#1}}

\titleformat{\section}{\normalfont\LARGE\bfseries\color{blacktextemphcolor}}%
{\tcbox[colback=blackpagecolor, colframe=blackpagecolor, coltext=whitetextdimmedcolor, on line, boxsep=0pt, left=4pt, right=4pt, top=4pt, bottom=4pt]{\textcolor{whitetextcolor}{\thesection}}}{0.4em}{}
\titleformat{\subsection}{\normalfont\Large\bfseries\color{blacktextemphcolor}}%
{\tcbox[colback=blackpagecolor, colframe=blackpagecolor, coltext=whitetextdimmedcolor, on line, boxsep=0pt, left=4pt, right=4pt, top=4pt, bottom=4pt]{\thesection.\textcolor{whitetextcolor}{\thesubsection}}}{0.4em}{}
\titleformat{\subsubsection}{\normalfont\large\bfseries\color{blacktextemphcolor}}%
{\tcbox[colback=blackpagecolor, colframe=blackpagecolor, coltext=whitetextdimmedcolor, on line, boxsep=0pt, left=4pt, right=4pt, top=4pt, bottom=4pt]{\thesection.\thesubsection.{\normalsize\selectfont\textcolor{whitetextcolor}{\thesubsubsection}}}}{0.4em}{}

\newcommand{\textitemize}{–\xspace}
\newcommand{\textitemizesecondary}{\textbullet\xspace}

\begin{document}
%% include 'Cover.tex'
\newpage
\section[Présentation générale]{Présentation générale de la spécialité et des parcours}
Ce document présente le programme national du B.U.T \VAR{data.name|le}
et complète l’annexe 1 de l’arrêté relatif aux programmes nationaux de
la licence professionnelle-bachelor universitaire de technologie.


%% for paragraph in data.introtxt
\VAR{paragraph|le}

%% endfor

%% for parc in data.parcours.values()
\subsection*{\VAR{parc.getCanonical(lower=False)|le} «~\VAR{parc.getName()|le}~»}
%% for paragraph in parc.introtxt
\VAR{paragraph|le}

%% endfor
%% endfor

\selectallparcours
\newpage
\section{Référentiel de formation}
\subsection{Tableaux horaires}
\selectallparcours
%% for sem in data.getSemestre()
%% for parcours in data.getParcours(data.getModule(semestre=sem).getParcoursBinBasic())
%% include 'Horaires.tex'
%% endfor
%% endfor
\newpage
\subsection{Coefficients}
%% for sem in data.getSemestre()
%% for parcours in data.getParcours(data.getModule(semestre=sem).getParcoursBinBasic())
%% include 'Coeffs.tex'
%% endfor
%% endfor
\newpage
\subsection{Interaction entre SAÉ et ressources}
%% for sem in data.getSemestre()
%% for parcours in data.getParcours(data.getModule(semestre=sem).getParcoursBinBasic())
%% include 'Flechage.tex'
%% endfor
%% endfor
\newpage
\subsection{Interaction entre SAÉ, ressources et apprentissages critiques}
%% for year in data.getYear()
%% for parcours in data.getParcours(data.getModule(year=year).getParcoursBinBasic())
%% include 'Matrice.tex'
%% endfor
%% endfor

\newpage
\subsection{Cadre général}
\subsubsection*{Les situations d’apprentissage et d’évaluation}
Les SAÉ permettent l’évaluation en situation de la compétence. Cette
évaluation est menée en correspondance avec l’ensemble des éléments
structurants le référentiel, et s’appuie sur la démarche portfolio, à
savoir une démarche de réflexion et de démonstration portée par la
personne elle-même.  Parce que cette démarche répond à une problématique
que l’on retrouve en milieu professionnel, une SAÉ est une tâche
authentique.

En tant qu’ensemble d’actions, la SAÉ nécessite de la part de la
personne qui la met en œuvre le choix, la mobilisation et la combinaison
de ressources pertinentes et cohérentes avec les objectifs ciblés.

L’enjeu d’une SAÉ est ainsi multiple :
\begin{itemize}
\item Participer au développement de la compétence ;
\item Soutenir l’apprentissage et la maîtrise des ressources ;
\item Intégrer l’autoévaluation ;
\item Permettre une individualisation des apprentissages.
\end{itemize}

Au cours des différents semestres de formation, la confrontation à
plusieurs SAÉ qui permettront de développer et de mettre en œuvre
chaque niveau de compétence ciblé dans le respect des composantes
essentielles du référentiel de compétences et en cohérence avec les
apprentissages critiques.

%% set services=["CJ", "CS", "GACO", "GEA", "GLT", "INFOCOM", "STID", "TD"]
Les SAÉ peuvent mobiliser des heures issues des
%% if data.id|upper in services
1800
%% else
2000
%% endif
h de formation et des 600 h de projet. Les SAÉ prennent la forme de
dispositifs pédagogiques variés, individuels ou collectifs, organisés
dans un cadre universitaire ou extérieur, tels que des ateliers, des
études, des challenges, des séminaires, des immersions au sein d’un
environnement professionnel, des stages, etc.

\subsubsection*{La démarche portfolio}%
%% for portfolio in data.portfolio.values()
\label{FICHE-\VAR{portfolio.id|le}}%
%% endfor

Nommé parfois portefeuille de compétences ou passeport professionnel, le
portfolio est un point de connexion entre le monde universitaire et le
monde socio-économique. En cela, il répond à l’ensemble des dimensions
de la professionnalisation de la personne inscrite en B.U.T. : de sa
formation à son devenir en tant que professionnel.

Le portfolio soutient donc le développement des compétences et
l’individualisation du parcours de formation.  Plus spécifiquement, le
portfolio offre la possibilité pour elle d’engager une démarche de
démonstration, de progression, d’évaluation et de valorisation des
compétences qu’il acquiert tout au long de son cursus.

Quels qu’en soient la forme, l’outil ou le support, le portfolio a pour
objectif de lui permettre d’adopter une posture réflexive et critique
vis-à-vis des compétences acquises ou en voie d’acquisition. Au sein du
portfolio, sa trajectoire de développement est documentée et argumentée
en mobilisant et analysant des traces, et ainsi en apportant des preuves
issues de l’ensemble de ses mises en situation professionnelle (SAÉ).

La démarche portfolio est un processus continu d’autoévaluation qui
nécessite un accompagnement par l’ensemble des membres de l’équipe
pédagogique. Ceux-ci guident la compréhension des éléments du
référentiel de compétences, ses modalités d’appropriation, les mises en
situation correspondantes et les critères d’évaluation.

\subsubsection*{Le projet personnel et professionnel}
Présent à chaque semestre de la formation et en lien avec les réflexions
de l’équipe pédagogique, le projet personnel et professionnel est un
élément structurant qui permet à la personne inscrite en B.U.T. d’agir
sur sa formation, d’en comprendre et de s’en approprier les contenus,
les objectifs et les compétences ciblées. Il assure également un
accompagnement de la personne dans sa propre définition d’une stratégie
personnelle et dans la construction de son identité professionnelle, en
cohérence avec les métiers et les situations professionnelles couverts
par la spécialité « \VAR{data.name|le} » et les parcours
associés. Enfin, le PPP la prépare à évoluer tout au long de sa vie
professionnelle, en lui fournissant des méthodes d’analyse et
d’adaptation aux évolutions de la société, des métiers et des
compétences.  Par sa dimension personnelle, le PPP vise à :
\begin{itemize}
\item Induire un questionnement sur son propre projet et son propre
  parcours de formation ;
\item Lui donner les moyens d’intégrer les codes du
  monde professionnel et socio-économique ;
\item L’aider à se définir et à se positionner ;
\item Le guider dans son évolution et son devenir ;
\item Développer sa capacité d’adaptation.
\end{itemize}
Au plan professionnel, le PPP permet :
\begin{itemize}
\item Une meilleure appréhension des objectifs de la formation, du
  référentiel de compétences et du référentiel de formation ;
\item Une connaissance exhaustive des métiers et perspectives
  professionnelles spécifiques à la spécialité et ses parcours ;
\item L’usage contextualisé des méthodes et des outils en lien avec la
  démarche de recrutement, notamment dans le cadre d’une recherche de
  contrat d’alternance ou de stage ;
\item La construction d’une identité professionnelle au travers des
  expériences de mise en situation professionnelle vécues pendant la
  formation.
\end{itemize}
Parce qu’ils participent tous deux à la professionnalisation et en cela
sont en dialogue, le PPP et la démarche portfolio ne doivent pourtant
pas être confondus. Le PPP répond davantage à un objectif
d’accompagnement qui dépasse le seul cadre des compétences à acquérir,
alors que la démarche portfolio répond fondamentalement à des enjeux
d’évaluation des compétences.


\subsection{Fiches SAÉ et Ressources}
%% for sem in data.getSemestre()
%% set ls=data.getModule(onlyType=['SAE'],semestre=sem)
%% set lr=data.getModule(onlyType=['RESS'],semestre=sem)
%% for ll,lname in [(ls,'SAÉ'),(lr,'Ressources')]

\textbf{Semestre \VAR{sem|le}, \VAR{lname}}\\
%% for l in ll
\makebox[\linewidth]{\quad \hyperref[FICHE-\VAR{l.id|le}]{\VAR{l.id|le}~\VAR{l.getName()|le}}\dotfill\pageref{FICHE-\VAR{l.id|le}}}\par
%% endfor
%% endfor
%% endfor
%% for sem in data.getSemestre()
%% set ls=data.getModule(onlyType=['SAE'],semestre=sem)
%% set lr=data.getModule(onlyType=['RESS'],semestre=sem)
%% for module in ls
%% include 'SAE.tex'
%% endfor
%% for module in lr
%% include 'Ressource.tex'
%% endfor
%% endfor

\section{Dispositions particulières}
Non-applicable.
\section{Référentiel d’évaluation}
Les dispositions relatives à l’évaluation sont décrites dans l’annexe 1
de l’arrêté relatif aux programmes nationaux de la licence
professionnelle-bachelor universitaire de technologie.
\newpage
\tableofcontents
\end{document}

