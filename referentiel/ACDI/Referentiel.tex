\documentclass[10pt]{article}

%% set margin=printer.margin
%% set layout=printer.layout
%% set color=printer.color
%% include 'packages.tex'

%% set complettres=['X','A','B','C','D','E','F']
%% set services=["CJ", "CS", "GACO", "GEA", "GLT", "INFOCOM", "STID", "TD"]
%% if data.id|upper in services
%% set dpttertiaire=True
%% else
%% set dpttertiaire=False
%% endif

\begin{document}
%% include 'Cover.tex'
\newpage
\section[Présentation générale]{Présentation générale de la spécialité et des parcours}
Ce document présente une proposition complète de programme national du
B.U.T \VAR{data.name|le} et complète les publications des annexes de
l'arrêté relatif aux programmes nationaux de la licence professionnelle
— bachelor universitaire de technologie.

%% for paragraph in data.introtxt
\VAR{paragraph|le}

%% endfor

%% for parc in data.parcours.values()
\subsection*{\VAR{parc.getCanonical(lower=False)|le} «~\VAR{parc.getName()|le}~»}
%% for paragraph in parc.introtxt
\VAR{paragraph|le}
%% endfor
%% endfor

\subsection[Architecture du diplôme]{Architecture du diplôme}
Le bachelor universitaire de technologie est défini par une spécialité
et un parcours. La spécialité \VAR{data.getName()|le} de bachelor
universitaire de technologie propose \VAR{data.parcours|length}
parcours.

Un parcours définit précisément un cursus de bachelor universitaire de technologie au sein d’une
spécialité donnée. Il vise un champ d’activité, une famille de métiers identifiés et répond à des
enjeux d’individualisation en lien avec le projet personnel et professionnel.

Il est certifié par 4 à 6 blocs de compétences, aussi dénommés « compétences finales » dans
l’approche par compétences et entendues comme des « savoirs agir complexes » mis en œuvre
dans un contexte professionnel et qui mobilisent et combinent des ressources acquises au cours du
cursus. Chaque bloc de compétences est décliné par niveau tout au long du parcours.

\subsection{Taille des groupes}
Les groupes sont de
%% if dpttertiaire
26
%% else
28
%% endif
personnes en Travaux Dirigés (TD) et
%% if dpttertiaire
13
%% else
14
%% endif
en Travaux Pratiques (TP).

Le référentiel de formation identifie les TP présentant un risque pour la sécurité des personnes et
nécessitant un encadrement particulier.

\subsection{Évaluation interne des formations}
Chaque département de l’IUT met en place un conseil de perfectionnement
conformément aux statuts de son établissement. Dans une logique
d’amélioration continue, le conseil de perfectionnement examine une fois
par an les indicateurs du bachelor universitaire de technologie de la
spécialité, notamment les résultats des évaluations des formations et
des enseignements par les personnes qui les ont suivis, les suivis de
cohortes, la qualité des stages et le suivi de l’insertion
professionnelle.

Le Conseil de l’IUT est ensuite informé de l’ensemble des évaluations
internes des départements.

\newpage
\section{Référentiel d'activités et de compétences}

\subsection{Les compétences}

%% include 'CompPanorama.tex'

\newpage
\subsection{Détail des parcours}
%% for parcours in data.parcours.values()|sort(attribute='id')
%% include 'Parcours.tex'
%% endfor


\selectallparcours
\newpage
\section{Référentiel de formation}
\subsection{Cadre général}
Le bachelor universitaire de technologie est organisé en 6 semestres
composés d’unités d’enseignement (UE) et chaque niveau de développement
des compétences se déploie sur les deux semestres d’une même année.

Les UE et les compétences sont mises en correspondance. Chaque UE se
réfère à une compétence finale et à un niveau de cette compétence. Elle
est nommée par le numéro du semestre et celui de la compétence finale.

Chaque unité d’enseignement est composée de deux éléments constitutifs :
\begin{itemize}
\item un pôle «~Ressources~», qui permet l’acquisition des connaissances et méthodes
fondamentales,
\item un pôle «~Situation d'apprentissage et d'évaluation~» (SAÉ) qui
  englobe les mises en situation professionnelle au cours desquelles la
  compétence se développe et à partir desquelles il sera fait la
  démonstration de l’acquisition de cette compétence dans la démarche
  portfolio.
\end{itemize}

\subsubsection{Les situations d'apprentissage et d'évaluation}
Les SAÉ permettent l'évaluation en situation de la compétence. Cette
évaluation est menée en correspondance avec l'ensemble des éléments
structurants le référentiel, et s'appuie sur la démarche portfolio, à
savoir une démarche de réflexion et de démonstration portée par la
personne elle-même.  Parce que cette démarche répond à une problématique
que l'on retrouve en milieu professionnel, une SAÉ est une tâche
authentique.

En tant qu'ensemble d'actions, la SAÉ nécessite de la part de la
personne qui la met en œuvre le choix, la mobilisation et la combinaison
de ressources pertinentes et cohérentes avec les objectifs ciblés.

L'enjeu d'une SAÉ est ainsi multiple :
\begin{itemize}
\item Participer au développement de la compétence ;
\item Soutenir l'apprentissage et la maîtrise des ressources ;
\item Intégrer l'autoévaluation ;
\item Permettre une individualisation des apprentissages.
\end{itemize}

Au cours des différents semestres de formation, la confrontation à
plusieurs SAÉ qui permettront de développer et de mettre en œuvre
chaque niveau de compétence ciblé dans le respect des composantes
essentielles du référentiel de compétences et en cohérence avec les
apprentissages critiques.

Les SAÉ peuvent mobiliser des heures issues des
%% if dpttertiaire
1800
%% else
2000
%% endif
h de formation et des 600 h de projet. Les SAÉ prennent la forme de
dispositifs pédagogiques variés, individuels ou collectifs, organisés
dans un cadre universitaire ou extérieur, tels que des ateliers, des
études, des challenges, des séminaires, des immersions au sein d'un
environnement professionnel, des stages, etc.

\subsubsection{Adaptation locale}
L’adaptation locale s’entend comme la définition par chaque IUT du contenu et des modalités des
enseignements. Elle ne peut pas modifier le référentiel de compétences et d’activités et définir,
notamment, de nouveaux niveaux de compétences ni de nouvelles compétences finales.

L’adaptation locale représente un tiers du volume global des heures d’enseignement, soit
%% if dpttertiaire
600
%% else
667
%% endif
heures d’enseignement pour une spécialité secondaire sur les 3 ans, ou
600 heures d’enseignement pour une spécialité tertiaire sur les 3
ans. Elle représente chaque année au maximum 40\% du volume horaire
d’enseignement de l’année hors projets tutorés.

\subsubsection{Compétences transversales et enjeux sociétaux}
L’acquisition des connaissances et compétences dans les secteurs
professionnels et les métiers visés permet d’acquérir aussi des
compétences transversales et ainsi de développer une pensée critique et
d’appréhender les concepts et les enjeux de développement durable, de
mondialisation, d’interculturalité et de transition écologique, de
responsabilité sociétale, d’éthique, notamment des problématiques liées
aux situations de handicap, à l’accessibilité et à la conception
universelle.

La formation intègre un volume d’enseignement d’expression-communication
et d’au moins une langue étrangère qui participe au développement d’une
culture communicationnelle et informationnelle ainsi qu’à la maîtrise
des techniques médiatiques associées, et adaptées notamment à
l’environnement professionnel de chaque spécialité.
\subsubsection{Passerelles et paliers d’orientation}
Une souplesse des dispositifs pédagogiques facilite l’intégration de
publics post-bac diversifiés ayant des acquis différents à l’entrée en
formation comme en cours de cursus. Elle permet également de lisser la
marche de début de cursus pour limiter les échecs en première année.

Dans chaque spécialité, les passerelles entrantes sont prévues sur les
semestres 3 et 5. Les IUT affichent le nombre de places disponibles pour
ces entrées latérales et réunissent sous la présidence du directeur, une
commission d’admission chargée d’étudier les demandes et de préciser le
contrat pédagogique de l’entrant.

Dans ce processus d’intégration en cours de cursus, une attention
particulière sera portée à l’accueil des titulaires du BTS et aux
personnes engagées dans les formations menant au diplôme national de
licence.

Des paliers d’orientation sont prévus en fin de S1, S2 et de S4
permettant la mise en œuvre de passerelles vers d’autres formations,
notamment licences, BTS ou écoles.

\subsubsection{Le projet personnel et professionnel}
Présent à chaque semestre de la formation et en lien avec les réflexions
de l'équipe pédagogique, le projet personnel et professionnel est un
élément structurant qui permet à la personne inscrite en B.U.T. d'agir
sur sa formation, d'en comprendre et de s'en approprier les contenus,
les objectifs et les compétences ciblées. Il assure également un
accompagnement de la personne dans sa propre définition d'une stratégie
personnelle et dans la construction de son identité professionnelle, en
cohérence avec les métiers et les situations professionnelles couverts
par la spécialité « \VAR{data.name|le} » et les parcours
associés. Enfin, le PPP la prépare à évoluer tout au long de sa vie
professionnelle, en lui fournissant des méthodes d'analyse et
d'adaptation aux évolutions de la société, des métiers et des
compétences.  Par sa dimension personnelle, le PPP vise à :
\begin{itemize}
\item Induire un questionnement sur son propre projet et son propre
  parcours de formation ;
\item Lui donner les moyens d'intégrer les codes du
  monde professionnel et socio-économique ;
\item L'aider à se définir et à se positionner ;
\item Le guider dans son évolution et son devenir ;
\item Développer sa capacité d'adaptation.
\end{itemize}
Au plan professionnel, le PPP permet :
\begin{itemize}
\item Une meilleure appréhension des objectifs de la formation, du
  référentiel de compétences et du référentiel de formation ;
\item Une connaissance exhaustive des métiers et perspectives
  professionnelles spécifiques à la spécialité et ses parcours ;
\item L'usage contextualisé des méthodes et des outils en lien avec la
  démarche de recrutement, notamment dans le cadre d'une recherche de
  contrat d'alternance ou de stage ;
\item La construction d'une identité professionnelle au travers des
  expériences de mise en situation professionnelle vécues pendant la
  formation.
\end{itemize}

\subsubsection{La démarche portfolio}%
Nommé parfois portefeuille de compétences ou passeport professionnel, le
portfolio est un point de connexion entre le monde universitaire et le
monde socio-économique. En cela, il répond à l'ensemble des dimensions
de la professionnalisation de la personne inscrite en B.U.T. : de sa
formation à son devenir en tant que professionnel.

Le portfolio soutient donc le développement des compétences et
l'individualisation du parcours de formation.  Plus spécifiquement, le
portfolio offre la possibilité pour elle d'engager une démarche de
démonstration, de progression, d'évaluation et de valorisation des
compétences qu'il acquiert tout au long de son cursus.

Quels qu'en soient la forme, l'outil ou le support, le portfolio a pour
objectif de lui permettre d'adopter une posture réflexive et critique
vis-à-vis des compétences acquises ou en voie d'acquisition. Au sein du
portfolio, sa trajectoire de développement est documentée et argumentée
en mobilisant et analysant des traces, et ainsi en apportant des preuves
issues de l'ensemble de ses mises en situation professionnelle (SAÉ).

La démarche portfolio est un processus continu d'autoévaluation qui
nécessite un accompagnement par l'ensemble des membres de l'équipe
pédagogique. Ceux-ci guident la compréhension des éléments du
référentiel de compétences, ses modalités d'appropriation, les mises en
situation correspondantes et les critères d'évaluation.

Parce qu'ils participent tous deux à la professionnalisation et en cela
sont en dialogue, le PPP et la démarche portfolio ne doivent pourtant
pas être confondus. Le PPP répond davantage à un objectif
d'accompagnement qui dépasse le seul cadre des compétences à acquérir,
alors que la démarche portfolio répond fondamentalement à des enjeux
d'évaluation des compétences.


\subsubsection{Stages}
Le stage contribue à la professionnalisation et à la validation des
compétences du Bachelor Universitaire de Technologie. Les stages sont
répartis selon le calendrier suivant : 8 à 12 semaines les 4 premiers
semestres ; 12 à 16 semaines la dernière année.

Les CPN décident de la durée et du positionnement des différentes
périodes de stages en respectant la limite de 22 à 26 semaines de
l’arrêté. Des dérogations pourront éventuellement être envisagées pour
les professions réglementées.

L’encadrement des stages est assuré par les membres de l’équipe
pédagogique en coordination avec l’organisme d’accueil. Cet encadrement
recouvre en particulier la validation des missions, le suivi régulier du
stagiaire et son évaluation.

L’encadrement du stage fait l’objet d’une reconnaissance par
l’établissement notamment au travers du référentiel national
d’équivalences horaires.

\subsubsection{Projets tutorés}
D’un volume total de 600 heures, les projets tutorés sont des axes
structurants de la professionnalisation en tant qu’ils participent de
l’acquisition des compétences du référentiel du Bachelor Universitaire
de Technologie et du parcours associé.  En cohérence avec l’approche par
compétences, les projets tutorés sont des éléments essentiels et
fondamentaux du pôle « Situation d’Apprentissage et d’Évaluation » (SAÉ)
des UE de chaque semestre.

Prenant la forme d’activités encadrées par les membres de l’équipe
pédagogique dont une partie issue du monde socio-économique, les 600h de
projets tutorés supposent donc une pédagogie innovante et adaptée qui
s’appuie sur un volume d’heures de formation à hauteur minimale de
\textbf{75 HETD par an et par groupe de TD}, en complément de celui des
1800 ou des 2000 heures d’enseignement selon la spécialité.

\subsubsection{Alternance}
L’alternance peut être réalisée sur l’ensemble de la formation. Elle
favorise l’insertion professionnelle.

Afin de tenir compte de l’acquisition de compétences en entreprise, les
maquettes de formation de chaque année en alternance, incluant les
projets tutorés, sont réduites de 15 à 25% du volume
horaire global de l’année. Cette diminution peut être appliquée sur les
enseignements encadrés comme sur les projets. Elle doit être répartie
sur l’ensemble des semestres du cursus.  Le référentiel de formation
définit pour chacune des spécialités la valeur du pourcentage de
réduction du volume horaire annuel dans la fourchette proposée.

Le suivi des alternants est une modalité pédagogique qui est définie par
le conseil de perfectionnement en accord avec les employeurs et prise en
compte pour les enseignants dans le cadre du référentiel des
équivalences horaires voté et appliqué par chaque établissement.
\subsubsection{Internationalisation}

Pour chaque spécialité des dispositifs d’ouverture à l’international
et/ou de sensibilisation à l’interculturalité sont mis en œuvre.

\subsubsection{Enseignement à Distance}
L’enseignement à distance peut être mis en œuvre, soit pour modifier les
modalités de travail en présentiel, soit pour remplacer l’enseignement
en présentiel. Dans tous les cas, l’enseignement à distance ne doit pas
alourdir les horaires d’enseignement reçus au-delà des 33h/semaine.

\subsection{Tableaux horaires}
\selectallparcours
Le volume horaire global (enseignement et projets tutorés, soit
%% if dpttertiarie
2400
%% else
2600
%% endif
\unskip) est distribué de manière homogène sur
les trois années, sans excéder chaque année une moyenne maximum de 33 heures par semaine.
Les 600 heures de projets tutorés sont réparties sur les trois années, avec chaque année un
minimum de 150 heures et un maximum de 250 heures ; ces heures sont clairement identifiées
dans les maquettes de formation et dans les emplois du temps afin de valoriser cette modalité
pédagogique et d’en assurer le déploiement.

La répartition horaire consacre au moins 
%% if dpttertiaire
40\% des heures d'étude (1800 h + 600 h projets)
%% else
50\% des heures d'étude (2000 h + 600 h projets)
%% endif
aux enseignements pratiques et aux mises en situation professionnelle.

%% for sem in data.getSemestre()
%% for parcours in data.getParcours(data.getModule(semestre=sem).getParcoursBinBasic())
%% include 'Horaires.tex'
%% endfor
%% endfor
\newpage
\subsection{Coefficients}
%% for sem in data.getSemestre()
%% for parcours in data.getParcours(data.getModule(semestre=sem).getParcoursBinBasic())
%% include 'Coeffs.tex'
%% endfor
%% endfor
\newpage
\subsection{Interaction entre SAÉ et ressources}
%% for sem in data.getSemestre()
%% for parcours in data.getParcours(data.getModule(semestre=sem).getParcoursBinBasic())
%% include 'Flechage.tex'
%% endfor
%% endfor
\newpage
\subsection{Interaction entre SAÉ, ressources et apprentissages critiques}
%% for year in data.getYear()
%% for parcours in data.getParcours(data.getModule(year=year).getParcoursBinBasic())
%% include 'Matrice.tex'
%% endfor
%% endfor

\newpage
\subsection{Fiches SAÉ et Ressources}
%% for sem in data.getSemestre()
%% set ls=data.getModule(onlyType=['SAE'],semestre=sem)
%% set lr=data.getModule(onlyType=['RESS'],semestre=sem)
%% for ll,lname in [(ls,'SAÉ'),(lr,'Ressources')]

\textbf{Semestre \VAR{sem|le}, \VAR{lname}}\\
%% for l in ll
\makebox[\linewidth]{\quad \hyperref[FICHE-\VAR{l.id|le}]{\VAR{l.id|le}~\VAR{l.getName()|le}}\dotfill\pageref{FICHE-\VAR{l.id|le}}}\par
%% endfor
%% endfor
%% endfor
%% for sem in data.getSemestre()
%% set ls=data.getModule(onlyType=['SAE'],semestre=sem)
%% set lr=data.getModule(onlyType=['RESS'],semestre=sem)
%% set lp=data.getModule(onlyType=['PORTFOLIO'],semestre=sem)
%% for module in ls
%% include 'SAE.tex'
%% endfor
%% for module in lp
%% include 'Portfolio.tex'
%% endfor
%% for module in lr
%% include 'Ressource.tex'
%% endfor
%% endfor

\section{Dispositions particulières}
\subsection{L'alternance}
Le diplôme de B.U.T Informatique, quand il est préparé en alternance,
s’appuie sur le même référentiel de compétences et le même
référentiel de formation mais le volume horaire global de chaque
semestre sera réduit de 15\% en première année, de 20\% en
deuxième année et de 15\% en troisième année.

\section[Référentiel d'évaluation]{Référentiel d'évaluation}

\subsection{Contrôle continu}
Les unités d'Enseignement (UE) sont acquises dans le cadre d'un contrôle continu intégral. Celui-ci
s'entend comme une évaluation régulière pendant la formation reposant sur plusieurs épreuves.

\subsection{Assiduité}
L'assiduité est un élément important du contrat pédagogique pour la
réussite de la personne inscrite en B.U.T. L'obligation d'assiduité à
toutes les activités pédagogiques organisées dans le cadre de la
préparation du diplôme national de bachelor universitaire de technologie
est indissociable de l'évaluation par contrôle continu intégral.

Le règlement intérieur adopté par le conseil de l'IUT propose à
l'établissement les modalités d'application de cette
obligation. Lorsqu'elles ont une incidence sur l'évaluation, elles sont
arrêtées par les CFVU de chaque établissement ou tout autre organe en
tenant lieu sur proposition du Conseil de l'IUT.

\subsection{Conditions de validation}
Le bachelor universitaire de technologie s'obtient soit par acquisition
de chaque unité d'enseignement constitutive, soit par application des
modalités de compensation. Le bachelor universitaire de technologie
obtenu par l'une ou l'autre voie confère la totalité des 180 crédits
européens.

Une unité d'enseignement est définitivement acquise et capitalisable dès
lors que la moyenne obtenue à l'ensemble «~pôle ressources~» et « SAÉ »
est égale ou supérieure à 10. L'acquisition de l'unité d'enseignement
emporte l'acquisition des crédits européens correspondants.

À l'intérieur de chaque unité d'enseignement, le poids relatif des
éléments constitutifs, soit des pôles «~ressources~» et « SAÉ », varie
dans un rapport de 40 à 60\%. En troisième année ce rapport peut
toutefois être apprécié sur l'ensemble des deux unités d'enseignement
d'une même compétence.

La validation des deux UE du niveau d'une compétence emporte la
validation de l'ensemble des UE du niveau inférieur de cette même
compétence.

\subsection{Compensation}
La compensation s'effectue au sein de chaque unité d'enseignement ainsi
qu'au sein de chaque regroupement cohérent d'UE.

Seules les UE se référant à un même niveau d'une même compétence finale
peuvent ensemble constituer un regroupement cohérent. Des UE se référant
à des niveaux de compétence finale différents ou à des compétences
finales différentes ne peuvent pas appartenir à un même regroupement
cohérent. Aucune UE ne peut appartenir à plus d'un regroupement
cohérent.  Au sein de chaque regroupement cohérent d'UE, la compensation
est intégrale. Si une UE n'a pas été acquise en raison d'une moyenne
inférieure à 10, cette UE sera acquise par compensation si et seulement
si la moyenne a été obtenue au regroupement cohérent auquel l'UE
appartient.

\subsection{Règles de progression}
La poursuite d'études dans un semestre pair d'une même année est de
droit pour tous. La poursuite d'études dans un semestre impair
est possible si et seulement s'il a été obtenu :
\begin{itemize}
\item la moyenne à plus de la moitié des regroupements cohérents d'UE ;
\item et une moyenne égale ou supérieure à 8 sur 20 à chaque
  regroupement cohérent d'UE.
\end{itemize}
La poursuite d'études dans le semestre 5 nécessite de plus la validation
de toutes les UE des semestres 1 et 2 dans les conditions de validation
des points 4.3 et 4.4, ou par décision de jury.

Durant la totalité du cursus conduisant au bachelor universitaire de
technologie, on peut être autorisé à redoubler une seule fois
chaque semestre dans la limite de 4 redoublements. La direction de l'IUT
peut autoriser un redoublement supplémentaire en cas de force majeure
dûment justifiée et appréciée par ses soins. Tout refus d'autorisation
de redoubler est pris après avoir entendu la personne à sa demande. Il
doit être motivé et assorti de conseils d'orientation.

\subsection{Jury}
Le jury présidé par la direction de l'IUT délibère souverainement à
partir de l'ensemble des résultats obtenus. Il se réunit chaque semestre
pour se prononcer sur la progression et la validation des unités
d'enseignement, l'attribution du diplôme universitaire de technologie au
terme de l'acquisition des 120 premiers crédits européens du cursus et
l'attribution de la licence professionnelle « bachelor universitaire de
technologie ».

\begin{minipage}{\linewidth}
  \tableofcontents
\end{minipage}
\end{document}

