%% for comp in data.comp.values()|sort(attribute='number')
%% set cn=comp.number|int
%% if cn<0
%% set cn=0
%% endif
%% if complettres|length <= cn
%% set cn=0
%% endif
%% set cl=complettres[cn]

\invisiblesubsubsection{Compétence \VAR{comp.number|le} : \VAR{comp.getName()|le}}{
\begin{tikzpicture}[remember picture, overlay,baseline=(cornersw.south),
  x=1mm,y=1mm,every node/.style={inner sep=0pt,outer sep=0pt}]
  \coordinate (start) at (0,0);
  \coordinate (NW) at (start-|current page.north west);
  \coordinate (NE) at (start-|current page.north east);
  \coordinate (NWi) at ($ (NW)+(\VAR{margin},0)+(2,0) $);
  \coordinate (NEi) at ($ (NE)+(-\VAR{margin},0)+(-2,0) $);
  \node[align=flush center,
  anchor=north west,
  text width=186mm,
  text=comp\VAR{cl}pcolor
  ] (title) at ($ (NWi)+(0,-2) $) {\normalfont\large\bfseries
    Compétence \VAR{comp.number|le} : \VAR{comp.getName()|le}
  };
  \node[anchor=north west,text width=186mm,
  text=comp\VAR{cl}pcolorB
  ] (activite) at ($ (title.south west)+(0,-6) $) {
    %% for p in comp.description
    {
      %% if loop.first
      \itshape
      %% else
      \small
      %% endif
      \VAR{p|le}
    }\par
    %% endfor
  };
  \node[anchor=north west,text width=100mm,
  text=comp\VAR{cl}pcolorB
  ] (compo) at ($ (activite.south west)+(0,0) $) {
    %% for p in comp.component
    %% if loop.first
    \begin{itemize}[topsep=0pt,label=\textitemize,leftmargin=1pc,labelsep=*]
      %% endif
    \item en \VAR{p|le}
      %% if loop.last
    \end{itemize}
    %% endif      
    %% endfor
  };
  \node[anchor=north east,text width=80mm,
  text=comp\VAR{cl}pcolorC
  ] (sitpro) at ($ (activite.south east)+(0,-6) $) {
    %% for p in comp.context
    %% if loop.first
    \textbf{Situations professionnelles}
    \begin{itemize}[topsep=0pt,label=\textitemize,leftmargin=1pc,labelsep=*]
      %% endif
    \item \VAR{p|le}
      %% if loop.last
    \end{itemize}
    %% endif      
    %% endfor
  };
  \node[fit=(compo)(sitpro)] (bloc) {};
  \coordinate (last) at ($ (NWi|-bloc.south)+(0,-6) $);
  %% for subcomp in comp.subcomp.values()
  %% set level=subcomp.level|int
  %% set gradientletters=['C','B','']
  %% set gradientletter=gradientletters[level-1]
  %% set levelstr=level|string
  \node[align=flush center,
  anchor=north west,
  text width=60mm,
  text=comp\VAR{cl}pcolor\VAR{gradientletter}
  ] (xsublevel\VAR{subcomp.level|le}) at (last) {
    \begin{hyphenrules}{nohyphenation}%
      \normalfont\large\bfseries
      \VAR{subcomp.getName()|le}
    \end{hyphenrules}\par
    \normalfont\itshape\normalsize\selectfont \VAR{subcomp.getParcours().getCanonical(short=true)|le}
  };
  \node[align=flush left,
  anchor=north west,
  text width=120mm,
  text=comp\VAR{cl}pcolor\VAR{gradientletter}
  ] (xsublevelac\VAR{subcomp.level|le}) at ($ (last)+(66,0) $) {\normalfont
    %% for ac in subcomp.ac.values()
    %% if loop.first
    \begin{description}[topsep=0pt,style=standard,leftmargin=!,labelwidth=\widthof{\textbf{AC 0}}]
      %% endif
    \item[AC \VAR{ac.num|le}]\VAR{ac.getName()|le}\par
      %% if loop.last
    \end{description}
    %% endif      
    %% endfor
  };
  \node[fit=(xsublevel\VAR{subcomp.level|le})(xsublevelac\VAR{subcomp.level|le})] (block\VAR{subcomp.level|le}) {};
  \node[fit=(xsublevel\VAR{subcomp.level|le})(xsublevel\VAR{subcomp.level|le}.south|-block\VAR{subcomp.level|le}.south)] (sublevel\VAR{subcomp.level|le}) {};
  \node[fit=(xsublevelac\VAR{subcomp.level|le})(xsublevelac\VAR{subcomp.level|le}.south|-block\VAR{subcomp.level|le}.south)] (sublevelac\VAR{subcomp.level|le}) {};
  \coordinate (last) at ($ (block\VAR{subcomp.level|le}.south west)+(0,-11) $);
  %% endfor


  \begin{pgfonlayer}{bg}
    \coordinate(nw) at ($ (title.north west)+(-2,2) $);
    \coordinate(se) at ($ (title.south east)+(2,-2) $);
    \node[
    fill=comp\VAR{cl}color,
    draw=\comp\VAR{cl}colord,
    fit=(nw) (se),
    blur shadow={shadow blur steps=5}%
    ] (toptitle) {};
    
    \coordinate(nw) at ($ (activite.north west)+(-2,2) $);
    \coordinate(se) at ($ (activite.south east)+(2,-2) $);
    \node[
    fill=comp\VAR{cl}colorB,
    fit=(nw) (se),
    blur shadow={shadow blur steps=5}%
    ] (bottomtitlea) {};
    \coordinate(nw) at ($ (compo.north west)+(-2,0) $);
    \coordinate(se) at ($ (compo.south east|-bloc.south)+(2,-2) $);
    \node[
    fill=comp\VAR{cl}colorB,
    fit=(nw) (se),
    blur shadow={shadow blur steps=5}%
    ] (bottomtitleb) {};
    \coordinate(nw) at ($ (activite.north west)+(-2,2) $);
    \coordinate(se) at ($ (activite.south east)+(2,-2) $);
    \node[
    fill=comp\VAR{cl}colorB,
    fit=(nw) (se),
    ] (bottomtitlec) {};

    \path [
    fill=comp\VAR{cl}colorB,
    draw=\comp\VAR{cl}colorBd,
    ] (bottomtitleb.south east)|-(bottomtitlea.south east)|-(bottomtitlea.north west)|-cycle;
    
    \coordinate(nw) at ($ (sitpro.north west)+(-2,2) $);
    \coordinate(se) at ($ (sitpro.south east|-bloc.south)+(2,-2) $);
    \node[
    fill=comp\VAR{cl}colorC,
    draw=\comp\VAR{cl}colord,
    fit=(nw) (se),
    blur shadow={shadow blur steps=5}%
    ] (bottomtitle) {};
    %% for subcomp in comp.subcomp.values()
    %% set level=subcomp.level|int
    %% set gradientletters=['C','B','']
    %% set gradientletter=gradientletters[level-1]
    %% set levelstr=level|string
    \coordinate (topnw) at ($ (block\VAR{subcomp.level|le}.north west)+(-2,
    %% if loop.first
    2
    %% else
    7
    %% endif
    ) $);
    \coordinate (topse) at ($ (block\VAR{subcomp.level|le}.south east-|sublevel\VAR{subcomp.level|le}.north east)+(2,
    %% if loop.last
    -2
    %% else
    -2
    %% endif
    ) $);
    \coordinate (topne) at (topnw-|topse);
    \coordinate (topsw) at (topnw|-topse);
    \coordinate (tops) at ($ .5*(topse)+.5*(topsw)+(0,
    %% if loop.last
    0
    %% else
    -5
    %% endif
    ) $);
    \coordinate (topn) at ($ .5*(topne)+.5*(topnw)+(0,
    %% if loop.first
    0
    %% else
    -5
    %% endif
    ) $);
    \fill [
    fill=comp\VAR{cl}color\VAR{gradientletter},
    draw=\comp\VAR{cl}color\VAR{gradientletter}d,
    ] (topnw)--(topn)--(topne)--(topse)--(tops)--(topsw)--cycle;
    \coordinate(nw) at ($ (sublevelac\VAR{subcomp.level|le}.north west|-topnw)+(-2,0) $);
    \coordinate(se) at ($ (sublevelac\VAR{subcomp.level|le}.south east|-topse)+(2,0) $);
    \node[
    fill=comp\VAR{cl}color\VAR{gradientletter},
    draw=\comp\VAR{cl}color\VAR{gradientletter}d,
    fit=(nw) (se),
    ] (toptitle) {};
    %% endfor
  \end{pgfonlayer}
  \coordinate (cornernw) at (NWi|-toptitle.north);
  \coordinate (cornersw) at ($ (NWi|-last)+(0,-4) $);
\end{tikzpicture}}%
\BLOCK{for subcomp in comp.subcomp.values()}\label{FICHE-\VAR{subcomp.id}}\BLOCK{endfor}



%% endfor

\bigskip
Une \textbf{compétence} est un «~\textbf{savoir-agir complexe}, prenant appui sur la
mobilisation et la combinaison efficaces d’une variété de ressources à
l’intérieur d’une famille de situations~»~(Tardif, 2006). Les ressources
désignent ici les savoirs, savoir-faire et savoir-être dont dispose un
individu, et qui lui permettent de mettre en œuvre la compétence.

Les \textbf{situations professionnelles} réfèrent aux contextes dans lesquels les
compétences sont mises en jeu. Ces situations varient selon la
compétence ciblée.

Les niveaux de chaque compétence forment peuvent, selon le type de
B.U.T. proposé, être constitutifs de un, plusieurs ou tous les parcours.
