% # context : data, utils, module
\newpage
%% set year=utils.semestre2year(module.sem)|le
%% set possiblecomps=module.getParcours().getCompLevel(year)|sort(attribute='id')
%% set complettres=['X','A','B','C','D','E','F']
%% set targets=module.target.values()

\invisiblesubsubsection{Portfolio \VAR{module.id|le}~\emph{\VAR{module.getShortname()|le}}}{%
  \begin{tikzpicture}[remember picture,overlay,baseline=(cornersw.south),
    x=1mm,y=1mm,every node/.style={inner sep=0pt,outer sep=0pt}]
    %% set header='Portfolio'
    %% set subheader=''
    %% set maintitle=module.getName()
    %% set arianeList=[]
    %% set arianeEnd=module.getShortname()
    %% for comp in targets
    %% if arianeList.append(['Compétence '+comp.number+' : '+comp.getShortname()])
    %% endif
    %% endfor
    %% include 'TitleFragment.tex'
    \node[anchor=north west,text width={\xmarkfive}] (titredescriptif) at ($ (lastwest)+(2,-2) $) {
      \sloppy\bfseries\large\color{blacktextemphcolor}\selectfont
      Descriptif détaillé
    };
    \coordinate (last) at (titredescriptif.south-|NWi);
    \node[anchor=north west,text width={\xmarkfive}] (descriptif) at ($ (last)+(2,0) $) {
      \setlength\topsep{0pt}\setlength\partopsep{0pt}\begin{flushleft}
        %% if module.description
        \textbf{En quoi consiste cette SAÉ ?}\\
        \VAR{module.description|le|join('\\par\n')}
        %% endif
        %% set precos=module.guideline
        %% set livrables=module.production
        %% if livrables|length>0
        \\
        \textbf{Quelles sont les productions de cette SAÉ ?}\\
        \begin{itemize}[nosep,topsep=0pt,label=\textitemize,leftmargin=1pc,labelsep=*]
          %% for s in livrables
        \item \VAR{s|le}
          %% endfor
        \end{itemize}
        %% endif
        %% if precos|length>0
        \textbf{Comment se fait le travail ?}\\
        La préconisation est : \VAR{precos|le}.
        %% endif
      \end{flushleft}
    };
    \coordinate (lastwest) at ($ (descriptif.south-|NWi)+(0,-4) $);
    
    % 
    % Bilan horaire
    % 
    
    \node[%
    inner sep=2mm,rounded corners=2mm,%
    line width=.5mm,draw=black,%
    text width={\xmarkfive},anchor=north east] (combobox) at (lasteast-|NEi) {%
      {\large\bfseries Cursus}\hfill S\VAR{module.sem|le}\\
      %% set hoursout=data.getHours(source=module)
      %% set hoursin=data.getHours(destination=module)
      %% if hoursin.sum(onlyType=['PROJ'])>0
      \textbf{Travail encadré (projet tutoré)}\dotfill
      \textcolor{blacktextemphcolor}{\bfseries\VAR{hoursin.sum(onlyType=['PROJ'])|hours}} PT%
      \\
      %% endif
      \textbf{Formation complémentaire}\dotfill%
      \textcolor{blacktextemphcolor}{\bfseries\VAR{hoursin.sum(onlyType=['TD'])|hours}} TD et 
      \textcolor{blacktextemphcolor}{\bfseries\samebox{\bfseries 00h}{\VAR{hoursin.sum(onlyType=['TP'])|hours}}} TP\\
      %% set allress=data.getModule(onlyType='RESS',semestre=module.sem)|sort(attribute='id')
      %% for ress in allress if ress in hoursin.sources() or ress in module.ress.values()
      %% if loop.first
      Lien\VAR{'s' if loop.length>1 else ''} avec les ressources :\hfill    \llap{\rule[2.1ex]{8em}{1pt}}
      \\
      %% endif
      \hyperref[FICHE-\VAR{ress.id|le}]{\VAR{ress.id|le}~\VAR{ress.getShortname()|le}}
      %% if hoursin.sum(source=ress)>0
      \dotfill
      \VAR{hoursin.sum(source=ress,onlyType=['TD'])|hours} TD et 
      \samebox{\bfseries 00h}{\VAR{hoursin.sum(source=ress,onlyType=['TP'])|hours}} TP
      %% endif
      \\
      %% endfor
      Cela représente un total (encadrement et formation confondus) de \textcolor{blacktextemphcolor}{\bfseries\VAR{hoursin.sum()|hours}}.
    };

    \coordinate (lasteast) at ($ (combobox.south-|NEi)+(0,-2) $);

    %% include 'CompFragment.tex'
    \begin{pgfonlayer}{bg}
      \node[inner sep=2mm,rounded corners=2mm,draw=blacktextcolor,line width=.5mm,fit=(descriptif) (titredescriptif)]{};
    \end{pgfonlayer}
    \node [fit=(lastwest)(lasteast)] (cornersw) {};
  \end{tikzpicture}
}\label{FICHE-\VAR{module.id|le}}
\selectparcours{\VAR{module.getParcours().getCanonical()|le}}
